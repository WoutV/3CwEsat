\documentclass[a4paper, 12pt, titlepage]{report}

%Taal: Nederlands ("Inhoudsopgave", "Hoofdstuk",...)
\usepackage[dutch]{babel}

%Geen nummering bij secties en hoofdstukkden
\setcounter{secnumdepth}{-1} 

\usepackage[dutch]{babel}
\begin{document}

\title{\textbf{Numerieke Modellering en Benadering}\\\textit{Practicum 1: Eigenwaardenproblemen}\\}
\author{De Wolf Peter\\ Vekemans Wout}

\date{\today}
\begin{titlepage}
	\maketitle
	\thispagestyle{empty}
\end{titlepage}

\newpage

\listoffigures

\newpage

\section{Inleiding}
In dit practicum onderzoeken we methoden voor het bepalen van eigenwaarden van vole matrices. In een eerste sectie beschouwen we enkele theoretische eigenschappen van de methoden. In een tweede sectie worden de convergentie-eigenschappen van da methoden onderzocht aan de hand van MATLAB-experimenten. In de derde en laatste sectie gaan we dieper in op \'e\'en van de methoden, namelijk de Jacobi-methode.
\section{Theoretische eigenschappen}

\subsection{Opgave 1}

\subsection{Opgave 2}

\section{Convergentie-experimenten}

\subsection{Opgave 4}
Een volle, symmetrische matrix die wordt gereduceerd naar Hessenberg vorm, is tridiagonaal na reductie. Dit zorgt ervoor dat de hoeveelheid rekenwerk enorm verminderd wordt. 

\subsection{Opgave 5}

\subsection{Opgave 6}

\subsection{Opgave 7}

\section{Alternatieve eigenwaardenalgoritmen}

\subsection{Opgave 8}

\subsection{Opgave 9}

\subsection{Opgave 10}

\end{document}
