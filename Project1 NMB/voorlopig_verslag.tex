\documentclass[a4paper, 12pt, titlepage]{report}

%Taal: Nederlands ("Inhoudsopgave", "Hoofdstuk",...)
\usepackage[dutch]{babel}

%Geen nummering bij secties en hoofdstukkden
\setcounter{secnumdepth}{-1} 

\usepackage[dutch]{babel}
\begin{document}

\title{\textbf{Numerieke Modellering en Benadering}\\\textit{Practicum 1: Eigenwaardenproblemen}\\}
\author{De Wolf Peter\\ Vekemans Wout}

\date{\today}
\begin{titlepage}
	\maketitle
	\thispagestyle{empty}
\end{titlepage}

\newpage

\listoffigures

\newpage

\section{Inleiding}
In dit practicum onderzoeken we methoden voor het bepalen van eigenwaarden van vole matrices. In een eerste sectie beschouwen we enkele theoretische eigenschappen van de methoden. In een tweede sectie worden de convergentie-eigenschappen van da methoden onderzocht aan de hand van MATLAB-experimenten. In de derde en laatste sectie gaan we dieper in op \'e\'en van de methoden, namelijk de Jacobi-methode.
\section{Theoretische eigenschappen}

\subsection{Opgave 1}

\subsection{Opgave 2}

\section{Convergentie-experimenten}

\subsection{Opgave 4}
Het uitvoeren van het $spy$-commando laat zien dat er geen enkel $non-zero$ element in de matrix zit. De uitvoering van het QR-algoritme op een matrix met die afmetingen (nog steeds relatief klein) zou zeer veel werk vragen. Na het reduceren tot Hessenberg vorm zien we dat alle elementen onder de eerste benedendiagonaal nul zijn geworden. Verdere analyse leert ons dat ook de elementen op $\epsilon_{mach}$ na gelijk zijn aan nul. Dit komt doordat de originele matrix symmetrisch was. Het uitvoeren van het QR-algoritme voor het vinden van eigenwaarden van een matrix vraagt veel minder werk voor matrix van tridiagonale vorm.

\subsection{Opgave 5}

\subsection{Opgave 6}

\subsection{Opgave 7}

\section{Alternatieve eigenwaardenalgoritmen}

\subsection{Opgave 8}

\subsection{Opgave 9}

\subsection{Opgave 10}

\end{document}
